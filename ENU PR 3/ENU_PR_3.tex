\newcommand{\institut}{Institut f\"ur Telekommunikationssysteme}
\newcommand{\fachgebiet}{Nachrichten\"ubertragung}
\newcommand{\veranstaltung}{Praktikum Nachrichten\"ubertragung}
\newcommand{\pdfautor}{Dirk Babendererde (321 836), Thomas Kapa (325 219)}
\newcommand{\autor}{Dirk Babendererde (321 836)\\ Thomas Kapa (325 219)}
\newcommand{\gruppe}{Gruppe:}
\newcommand{\betreuer}{Betreuer: Lieven Lange}


\newcommand{\pdftitle}{Nachrichtenuebertragung\ Praktikum\ 03}
\newcommand{\prototitle}{Praktikum 03 \\ Statistische Nachrichtentheorie}

\input{../../packages/tu_header_8}


% \lstlistoflistings
\definecolor{darkgray}{rgb}{0.95,0.95,0.95}
\definecolor{darkolivegreen}{HTML}{01a801}
\definecolor{functionsBlue}{HTML}{32b9b9}
\definecolor{variableRed}{rgb}{1,0,0}
\definecolor{stringBrown}{HTML}{bc8e8e} % f geht nicht

\lstset{
        %\lstset{extendedchars=true} % Umlaute an der richtigen stelle und nicht am Anfang ausgeben
        %basicstyle=\footnotesize\ttfamily,
        basicstyle=\small,
        %
        inputencoding=utf8,
        %
        tabsize=4,
        showspaces=false,
        showtabs=false,
        showstringspaces=true, % no special string spaces
        %
        backgroundcolor=\color{darkgray}, % background
        stringstyle=\color{stringBrown}\fseries, % Strings
        keywordstyle=\color{functionsBlue}\bfseries, % keywords Blau
        identifierstyle=\color{variableRed}, % variablen
        commentstyle=\color{darkolivegreen}, %  comments
        %
        breaklines=true,
        %
        numbers=left,
        numberstyle=\tiny,
        stepnumber=1,
        numbersep=7pt,
        %
        frame=single,
        columns=flexible,
        %
        xleftmargin=-2cm,
        xrightmargin=-1.5cm,
        %
        language=Matlab,
}
% enables UTF-8 in source code: (dirty, dirty hack)
\lstset{literate=
    %Deutsch
    {ä}{{\"a}}1 {ö}{{\"o}}1 {ü}{{\"u}}1 {Ä}{{\"A}}1 {Ö}
    {{\"O}}1 {Ü}{{\"U}}1 {ß}{\ss}1
    %Türkisch
    {â}{{\^{a}}}1 {Â}{{\^{A}}}1 {ç}{{\c{c}}}1 {Ç}{{\c{C}}}1 {ğ}{{\u{g}}}1 {Ğ}{{\u{G}}}1 {ı}{{\i}}1 {İ}{{\.{I}}}1 {ö}{{\"o}}1 {Ö}{{\"O}}1 {ş}{{\c{s}}}1
    {Ş}{{\c{S}}}1 {ü}{{\"u}}1 {Ü}{{\"U}}1
    %Polish
    {ą}{{\k{a}}}1 {ć}{{\'c}}1 {ę}{{\k{e}}}1 {ł}{{\l{}}}1 {ń}{{\'n}}1 {ó}{{\'o}}1 {ś}{{\'s}}1 {ż}{{\.z}}1 {ź}{{\'z}}1 {Ą}{{\k{A}}}1 {Ć}{{\'C}}1
    {Ę}{{\k{E}}}1 {Ł}{{\L{}}}1 {Ń}{{\'N}}1 {Ó}{{\'O}}1 {Ś}{{\'S}}1 {Ż}{{\.Z}}1 {Ź}{{\'Z}}1
    %Spanish
    {á}{{\'a}}1 {é}{{\'e}}1 {í}{{\'i}}1 {ó}{{\'o}}1 {ú}{{\'u}}1 {ñ}{{\~n}}1
}

%     \lstinputlisting{./praktikum6.sce}



%---------------------------------------------------------------------
%---------------------------------------------------------------------
%---------------------------------------------------------------------


\section{Vorbereitungsaufgaben}
\begin{quote}
    \hspace{-2em}
    \subsection{AM}
        
    \begin{quote}
        
        Da alle Werte eines analogen Signals $\geq 0$ sein Müssen um sie mittels AM mit Träger übertragen zu können, haben wir für
        diesen Versuch mit Matlab Cosinus-, Dreieck- und Rechtecksignale folgenden Eigenschaften erzeugt:
        
        
        \begin{equation*}
        \begin{split}
        \\
                 0 &\leq u(t) \leq 2.0
        \\
                 f &= \si{100}{Hz}
        \\
            \alpha &= 0.5
        \\
                 T &= \si{2}{s}
        \\
               f_T &= \si{1}{MHz}
        \\
        \end{split}
        \end{equation*}
        
        
        Außerdem haben wir noch ein cosinus-Trägersignal mit \si{2}{kHz} erzeugt. Auf dieses Trägersignal haben wir dann, zum
        späteren vergleich, die erzeugten Basisbandsignale raufmoduliert.
        
    \end{quote}
    
    
    
    
    \subsection{FM}
    \begin{quote}
        
        
        Zur Vorbereitung der FM-Modulation haben wir ein Cosinussignal mit folgenden Eigenschaften simuliert.
        
        \begin{equation*}
        \begin{split}
        \\
                 0 &\leq u(t) \leq 2.0
        \\
                 f &= \si{100}{Hz}
        \\
                 T &= \si{0,5}{s}
        \\
               f_T &= \si{1}{MHz}
        \\
        \end{split}
        \end{equation*}
        
        
        Dieses Trägersignal haben wir dann mit einem weiteren cosinus ($f_u = 1kHz$ und $A_u = 1V$) nach der folgenden Formel
        moduliert:
        
        \begin{equation*}
        \begin{split}
        \\
                u_m(t) &= A \p cos(\varphi(t))
        \\
            \varphi(t) &= 2 \pi f_c t + K_{FM} \p \int\limits_{-\infty}^t u(\tau) \ d\tau  
        \\
        \end{split}
        \end{equation*}
        
        Von diesem Modulierten Signal ($u_m$) haben wir noch, zum späteren Vergleich, mit Hilfe von Matlab das Amplitudenspektrum
        bestimmt.
        
        
    \end{quote}
    
    
    \subsection{Theorie zur FM-Demodulation}
    \begin{quote}
        Wir haben im praktikum die FM-PFM-Umwandlung als Methode zur FM-Demodulation genutzt, da es ein relativ einfaches
        Verfahren ist. Hierbei wird das Signal zunächst durch einen Comparator in eine polare Rechteckfolge umgewandelt. Als
        Referenz für den Comparator wird \si{0}{\volt} eingestellt, damit jede positive Halbwelle des modulierten Signals zu
        einem positiven Rechteck wird und jede negative zu einem Negativen.\\
        Anschließend wird das Signal noch durch einen Twin Pulse Generator geführt, der aus jeder steigenden Flanke ein
        Rechteckimpuls macht.\\
        Wofür der Twin Pulse Generator gebraucht wird ist leider weder aus dem Skrip ersichtlich, noch konnte es uns im Praktikum
        einleuchtend erklärt werden. Er wird wohl ``wegen irgendwelcher Feedback-Sachen oder so''\footnotemark gebraucht.\\
        Zuletzt wird das Signal noch tiefpass-gefiltert, um aus der Häufigkeit der Pulse wieder das Analoge ausgangssignal zu
        machen.
        \footnotetext{Aussage des Tutors im NUE-Praktikum am 16.05.2012}
        
    \end{quote}
    
    
\end{quote}

%--------------------------------------------------------------------
%--------------------------------------------------------------------


\section{1 Amplitudenmodulation}
\begin{quote}
    
    
    \subsection{Labordurchführung}
    \begin{quote}
      Das mit Hilfe von Amplitudenmodulation zu übertragende Signal soll ein
      Sinussignal mit Amplitude von 1 V, einer Frequenz von 100 Hz, 
      mittelwertfrei sein und von Funktionsgenerator geliefert werden.
      Um das Signal von negativen Werten zu befreien (siehe Vorbereitung) wird
      das Signal mit Hilfe des Adder-Moduls und der variablen DC Voltage Quelle
      um ein Volt angehoben. Da das Adder-Modul den Ausgang invertiert, wird der
      Ausgang des ersten Adders auf einen zweiten Adder gegeben und der zweite
      Eingang auf Masse gelegt, um das Signal zu invertieren. //
      Um Trägersignal und das zu übertragende Signal zu überlagern wird das
      Multiplier-Modul verwendet. Dabei liefert das Master-Signal-Modul ein 2
      kHz Sinussignal, das als Trägersignal dient, das mit dem zu übertragenden
      Signal multipliziert wird.
        
    \end{quote}
    
    
    
    
    
    \subsection{Auswertung}
    \begin{quote}
        
        
        
        es ließ sich kein unterschied zwischen dem Signal nach dem Comparator und dem nach dem Twin Pulse Generator erkennen
    \end{quote}
    
\end{quote}


\section{2}
\begin{quote}
    
    
    \subsection{Labordurchführung}
    \begin{quote}
        
        
    \end{quote}
    
    
    
    
    
    \subsection{Auswertung}
    \begin{quote}
        
    \end{quote}
    
\end{quote}



% \begin{quote}
%     \lstinputlisting[
%         caption={Scilab-script},
%         label=lst:scilab]
%         {./Scilab/Motor.sce}
%         
% \end{quote}

%--------------------------------------------------------------------
%--------------------------------------------------------------------
% \begin{thebibliography}{999}
% 
% \bibitem{Boris}Boris Henckell: Ein Paar sachen geklaut.. ähhh inspirationen geholt
% \href{http://www.krachler.com/fileadmin/user_upload/arbeiten/Reglersynthese_Christian_Krachler.pdf}{Reglersynthese nach dem Frequenzkennlinienverfahren}, S16, S22, 08.05.2012
% 
% 
% %Name, Vorname.; evtl. Name2, Vorname2.: Titel des Dokumentes
% %oder Buches, Zeitschrift/Verlag/URL (Auflage, Erscheinungsort, -jahr), ggf. Seitenzahlen
% %\bibitem [Wiki10] {DigitaleMesskette2} \url{www.wikipedia.org}, Zugriff 22.03.2010
% \end{thebibliography}


\end{document}
