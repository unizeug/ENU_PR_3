\newcommand{\institut}{Institut f\"ur Telekommunikationssysteme}
\newcommand{\fachgebiet}{Nachrichten\"ubertragung}
\newcommand{\veranstaltung}{Praktikum Nachrichten\"ubertragung}
\newcommand{\pdfautor}{Dirk Babendererde (321 836), Thomas Kapa (325 219)}
\newcommand{\autor}{Dirk Babendererde (321 836)\\ Thomas Kapa (325 219)}
\newcommand{\gruppe}{Gruppe:}
\newcommand{\betreuer}{Betreuer: Lieven Lange}


\newcommand{\pdftitle}{Nachrichten\"ubertragung\ Praktikum\ 02}
\newcommand{\prototitle}{Praktikum 02 \\ Statistische Nachrichtentheorie}

\input{../../packages/tu_header_8}


% \lstlistoflistings
\definecolor{darkgray}{rgb}{0.95,0.95,0.95}
\definecolor{darkolivegreen}{HTML}{01a801}
\definecolor{functionsBlue}{HTML}{32b9b9}
\definecolor{variableRed}{rgb}{1,0,0}
\definecolor{stringBrown}{HTML}{bc8e8e} % f geht nicht

\lstset{
        %\lstset{extendedchars=true} % Umlaute an der richtigen stelle und nicht am Anfang ausgeben
        %basicstyle=\footnotesize\ttfamily,
        basicstyle=\small,
        %
        inputencoding=utf8,
        %
        tabsize=4,
        showspaces=false,
        showtabs=false,
        showstringspaces=true, % no special string spaces
        %
        backgroundcolor=\color{darkgray}, % background
        stringstyle=\color{stringBrown}\fseries, % Strings
        keywordstyle=\color{functionsBlue}\bfseries, % keywords Blau
        identifierstyle=\color{variableRed}, % variablen
        commentstyle=\color{darkolivegreen}, %  comments
        %
        breaklines=true,
        %
        numbers=left,
        numberstyle=\tiny,
        stepnumber=1,
        numbersep=7pt,
        %
        frame=single,
        columns=flexible,
        %
        xleftmargin=-2cm,
        xrightmargin=-1.5cm,
        %
        language=Matlab,
}
% enables UTF-8 in source code: (dirty, dirty hack)
\lstset{literate=
    %Deutsch
    {ä}{{\"a}}1 {ö}{{\"o}}1 {ü}{{\"u}}1 {Ä}{{\"A}}1 {Ö}
    {{\"O}}1 {Ü}{{\"U}}1 {ß}{\ss}1
    %Türkisch
    {â}{{\^{a}}}1 {Â}{{\^{A}}}1 {ç}{{\c{c}}}1 {Ç}{{\c{C}}}1 {ğ}{{\u{g}}}1 {Ğ}{{\u{G}}}1 {ı}{{\i}}1 {İ}{{\.{I}}}1 {ö}{{\"o}}1 {Ö}{{\"O}}1 {ş}{{\c{s}}}1
    {Ş}{{\c{S}}}1 {ü}{{\"u}}1 {Ü}{{\"U}}1
    %Polish
    {ą}{{\k{a}}}1 {ć}{{\'c}}1 {ę}{{\k{e}}}1 {ł}{{\l{}}}1 {ń}{{\'n}}1 {ó}{{\'o}}1 {ś}{{\'s}}1 {ż}{{\.z}}1 {ź}{{\'z}}1 {Ą}{{\k{A}}}1 {Ć}{{\'C}}1
    {Ę}{{\k{E}}}1 {Ł}{{\L{}}}1 {Ń}{{\'N}}1 {Ó}{{\'O}}1 {Ś}{{\'S}}1 {Ż}{{\.Z}}1 {Ź}{{\'Z}}1
    %Spanish
    {á}{{\'a}}1 {é}{{\'e}}1 {í}{{\'i}}1 {ó}{{\'o}}1 {ú}{{\'u}}1 {ñ}{{\~n}}1
}

%     \lstinputlisting{./praktikum6.sce}



%---------------------------------------------------------------------
%---------------------------------------------------------------------
%---------------------------------------------------------------------

\section{Vorbereitungsaufgaben}
\begin{quote}
    \hspace{-2em}
    \subsection{Varianz von gleichverteiltem weißem Rauschen}
    Berechnet die Varianz von gleichverteiltem weißem Rauschen N mit der Verteilungsdichtefunktion.\\
    \TODO{Verteilungsdichtefunktion}
    \begin{quote}
    
    \end{quote}
      
\end{quote}

%--------------------------------------------------------------------
%--------------------------------------------------------------------


\section{Labordurchführung}
\begin{quote}
    
    \subsection{Simulation eines Zufallsexperiments mit MatLab}
    \begin{quote}
        
        \subsubsection{Fairer Würfel?}
        Der MatLab-Code aus der Vorgabedatei Aufgabe1.m simuliert die 10.000-fache Durchführung eines Würfelexperiments. Ergänzt
        den Code, so dass aus dem einfachen Histogramm die geschätzte pdf des Experiments wird. Handelt es sich um einen fairen
        Würfel?
        
        \begin{quote}
            \begin{figure}[H]
            \centering
                \includegraphics[scale=0.7, trim = 20mm 80mm 20mm 90mm, clip]{Bilder/fairer_wuerfel}
                    \caption{Verteilungsdichtefunktion}
                    \label{fig:fairer_wuerfel}
            \end{figure}
            
            Wie in der Verteilungsdichtefunktion erkennbar ist, haben alle Seiten des Würfels die Gleiche Wahrscheinlichkeit.
            Somit ist der Würfel fair.
            
            \lstinputlisting[
                caption={Matlab-script},
                label=lst:Matlab]
                {./Matlab/Aufgabe1_1.m}
                
        \end{quote}
        
        
        \subsubsection{Summe aus Zwei Würfeln}
        Simuliert nun ein Experiment mit zwei Würfeln, bei dem nach jedem Wurf die Summe der Augenzahlen gebildet wird. Bestimmt
        auch für diesen Fall die pdf.
        
        \begin{quote}
            \begin{figure}[H]
            \centering
                \includegraphics[scale=0.7, trim = 20mm 80mm 20mm 90mm, clip]{Bilder/A1_2}
                    \caption{Verteilungsdichtefunktion}
                    \label{fig:A1_2}
            \end{figure}
            
            Auch diese Verteilungsdichtefunktion ist wie erwartet. 
            Die Logische Erklärung hierfür ist, dass es für das Ereignis 7 am meisten Kombiinationsmäglichkeiten
            der beiden Würfel gibt und es somit das wahrscheinlichste Ereignis ist.\\
            Auch die Theorie sehen wir hier bestätigt. Die Verteilungsdichtefunktion des Experiments mit zwei würfeln (ein
            Dreieck) ist die Faltung zweier Verteilungsdichtefunktionen des selben Experiments (Rechteck).
            
            
            \lstinputlisting[
                caption={Matlab-script},
                label=lst:Matlab]
                {./Matlab/Aufgabe1_2.m}
                
        \end{quote}
        
        
        
        \subsubsection{Zentraler Grenzwertsatz der Statistik}
        Laut zentralem Grenzwertsatz der Statistik, ergibt sich bei Aufsummierung von Zufallsvariablen mit
        weitgehend beliebigen Verteilungsfunktion als Ergebnis eine Zufallsvariable mit Gaußverteilung.
        Dieser Zusammenhang wird in Abbildung 1 verdeutlicht. Versucht den Grenzwertsatz nachzuvollziehen, indem ihr schrittweise
        erst 10 dann 100 und schließlich 1000 Realisierungen des Würfelexperiments additiv überlagert und die pdfs bestimmt.

        
        \begin{center}
        \begin{tabular}{ll}
        
        \hspace{-5cm}
            \begin{minipage}{0.6\textwidth}
                
                \begin{figure}[H]
                    \label{fig:funktion0alpha}
                    \includegraphics[scale=0.7, trim = 20mm 80mm 20mm 90mm, clip]{Bilder/fairer_wuerfel}
                    \caption{Verteilungsdichtefunktion: ein Würfel}
                \end{figure}
        
            \end{minipage}
        
            \begin{minipage}{0.6\textwidth}
                \begin{figure}[H]
                    \label{fig:pico_funktion0alpha}
                    \includegraphics[scale=0.7, trim = 20mm 80mm 20mm 90mm, clip]{Bilder/A1_2}
                    \caption{Verteilungsdichtefunktion: zwei Würfel}
                \end{figure}
        
            \end{minipage}
        
        \end{tabular}
        \end{center}

        \begin{center}
        \begin{tabular}{ll}
        
        \hspace{-5cm}
            \begin{minipage}{0.6\textwidth}
                
                \begin{figure}[H]
                    \label{fig:funktion0alpha}
                    \includegraphics[scale=0.7, trim = 20mm 80mm 20mm 90mm, clip]{Bilder/A1_3_10}
                    \caption{Verteilungsdichtefunktion: 10 Würfel}
                \end{figure}
        
            \end{minipage}
        
            \begin{minipage}{0.6\textwidth}
                \begin{figure}[H]
                    \label{fig:pico_funktion0alpha}
                    \includegraphics[scale=0.7, trim = 20mm 80mm 20mm 90mm, clip]{Bilder/A1_3_100}
                    \caption{Verteilungsdichtefunktion: 100 Würfel}
                \end{figure}
        
            \end{minipage}
        
        \end{tabular}
        \end{center}


        \begin{figure}[H]
        \centering
            \includegraphics[scale=0.7, trim = 20mm 80mm 20mm 90mm, clip]{Bilder/A1_3_1000}
                \caption{Verteilungsdichtefunktion: 1000 Würfel}
                \label{fig:A1_3_1000}
        \end{figure}
    


        \begin{quote}
            

            
            Es ist zu beobachten, dass je mehr würfel geworfen werden, desto mehr nähert sich die Verteilungsdichtefunktion der
            Gaußverteilung an.\\
            Die Löcher in den Plots entstehen dadurch, dass Würfelergebnisse immer nur diskrete Werte sind und somit nicht jeder
            Bin Belegt ist. Die nicht ganz exakte form könnte man mit noch mehr Würfen optimieren.
            
            
            \lstinputlisting[
                caption={Matlab-script},
                label=lst:Matlab]
                {./Matlab/Aufgabe1_3.m}
            
        \end{quote}
        
        
    \end{quote}

    \subsection{Rauschbehafteter Übertragungskanal}
    
    Im Folgenden sollen aus bekanntem Sende- und Empfangssignal Dämpfung und Rauschcharakteristik
    eines Übertragungskanals bestimmt werden. Es darf angenommen werden, dass die Dämpfung des
    Systems sich lediglich linear auf die Amplitude des Sendesignals auswirkt.
    
    \begin{quote}
        
        \subsubsection{-20dB Rauschen}
        \begin{quote}
            
            
        \begin{center}
        \begin{tabular}{ll}
        
        \hspace{-16.5em}
            \begin{minipage}{0.6\textwidth}
                
                \begin{figure}[H]
                    \label{fig:rau20}
                    \includegraphics[scale=0.7, trim = 20mm 80mm 20mm 90mm, clip]{Bilder/rau20}
                    \caption{Rauschanteil des Signals mit -20dB Rauschen}
                \end{figure}
        
            \end{minipage}
        
            \begin{minipage}{0.6\textwidth}
                \begin{figure}[H]
                    \includegraphics[scale=0.7, trim = 15mm 80mm 20mm 90mm, clip]{Bilder/hist20}
                    \caption{Verteilungsdichtefunktion des Rauschens}
                    \label{fig:hist20}
                \end{figure}
            
            \end{minipage}
        
        \end{tabular}
        \end{center}
            
            \vspace{2em}
            
            \begin{equation*}
            \begin{split}
                 SNR = -1.9448
            \end{split}
            \end{equation*}
            
            Das Rauschen ist relativ gering mit ca. $ \pm2,5$ Volt. Es ist, wie in Abbildung \ref{fig:hist20}, eher Gauß- als
            Laplaceverteilt.
            
        \end{quote}
        
        
        \subsubsection{-6dB Rauchen}
        \begin{quote}
        \begin{center}
        \begin{tabular}{ll}
        
        \hspace{-16.5em}
            \begin{minipage}{0.6\textwidth}
                
                \begin{figure}[H]
                    \label{fig:funktion0alpha}
                    \includegraphics[scale=0.7, trim = 20mm 80mm 20mm 90mm, clip]{Bilder/rau6}
                    \caption{Rauschanteil des Signals mit -6dB Rauschen}
                \end{figure}
                
            \end{minipage}
            
            \begin{minipage}{0.6\textwidth}
                \begin{figure}[H]
                    \includegraphics[scale=0.7, trim = 15mm 80mm 20mm 90mm, clip]{Bilder/hist6}
                    \caption{Verteilungsdichtefunktion des Rauschens}
                    \label{fig:hist6}
                \end{figure}
                
            \end{minipage}
            
        \end{tabular}
        \end{center}
            
            \vspace{2em}
            \begin{equation*}
            \begin{split}
                 SNR = 5.5350
            \end{split}
            \end{equation*}
            
            Das Rauschen bei -6 dB ist mit ca. $\pm 4$ Volt schon deutlich größer. Die Verteilungsdichtefunktion \ref{fig:hist6}
            ist noch eindeutiger eine Gaußverteilung.
            
            

        \end{quote}

        \subsubsection{0dB Rauschen}
        \begin{quote}
        \begin{center}
        \begin{tabular}{ll}
        
        \hspace{-16.5em}
            \begin{minipage}{0.6\textwidth}
                
                \begin{figure}[H]
                    \label{fig:funktion0alpha}
                    \includegraphics[scale=0.7, trim = 20mm 80mm 20mm 90mm, clip]{Bilder/rau0}
                    \caption{Rauschanteil des Signals mit 0dB Rauschen}
                \end{figure}
        
            \end{minipage}
        
            \begin{minipage}{0.6\textwidth}
                \begin{figure}[H]
                    \includegraphics[scale=0.7, trim = 15mm 80mm 20mm 90mm, clip]{Bilder/hist0}
                    \caption{Verteilungsdichtefunktion des Rauschens}
                    \label{fig:hist0}
                \end{figure}
        
            \end{minipage}
        
        \end{tabular}
        \end{center}
            
            \vspace{2em}
            
            \begin{equation*}
            \begin{split}
                 SNR = 17.7169
            \end{split}
            \end{equation*}
            
            Der Rauschanteil ist bei 0dB Rauschen mit ca. $\pm9$ Volt schon deutlich Größer als zuvor. Die Verteilung gleicht
            hier, bis auf einen kleinen Ausreißer, fast perfekt einer Gaußverteilung.
            
        \end{quote}

            \vspace{4em}

        Die SNR Werte der drei Rauschsignale scheinen, zumindest über unsere drei Messungen, ca. den folgenden Linearen
        zusammenhang zu haben.
        
        \begin{equation*}
    	\begin{split}
    		\mathrm{SNR \ [dB]} = \mathrm{Rauchen \ [dB]} - 2 
    	\end{split}
        \end{equation*}
        
        \newpage
        \lstinputlisting[
            caption={Matlab-script},
            label=lst:Matlab]
            {./Matlab/Aufgabe2.m}
        
        \lstinputlisting[
            caption={Matlab-script},
            label=lst:Matlab]
            {./Matlab/SNR.m}
        
    \end{quote}
    
    
    
\end{quote}





% \begin{quote}
%     \lstinputlisting[
%         caption={Scilab-script},
%         label=lst:scilab]
%         {./Scilab/Motor.sce}
%         
% \end{quote}

%--------------------------------------------------------------------
%--------------------------------------------------------------------
% \begin{thebibliography}{999}
% 
% \bibitem{Boris}Boris Henckell: Ein Paar sachen geklaut.. ähhh inspirationen geholt
% \href{http://www.krachler.com/fileadmin/user_upload/arbeiten/Reglersynthese_Christian_Krachler.pdf}{Reglersynthese nach dem Frequenzkennlinienverfahren}, S16, S22, 08.05.2012
% 
% 
% %Name, Vorname.; evtl. Name2, Vorname2.: Titel des Dokumentes
% %oder Buches, Zeitschrift/Verlag/URL (Auflage, Erscheinungsort, -jahr), ggf. Seitenzahlen
% %\bibitem [Wiki10] {DigitaleMesskette2} \url{www.wikipedia.org}, Zugriff 22.03.2010
% \end{thebibliography}


\end{document}
